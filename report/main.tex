% Formatage général
\documentclass[a4paper,11pt]{article}
\usepackage[T1]{fontenc}
\usepackage{helvet}
\usepackage[english]{babel}
\usepackage{csquotes}
\MakeOuterQuote{"}

% Dimensions
\usepackage[margin=2cm]{geometry}
\setlength{\parindent}{0pt}

% Couleurs
\usepackage{xcolor}
\definecolor{darkpowderblue}{rgb}{0.0, 0.2, 0.6}
\definecolor{gamboge}{rgb}{0.89, 0.61, 0.06}
\definecolor{ece}{RGB}{0, 122, 123}
\definecolor{verylightgray}{RGB}{240, 240, 240}

% Code listing
\usepackage{listings}
\usepackage{algorithm}
\usepackage{algpseudocode}

\lstset{
    language=C,
    basicstyle=\ttfamily\small,
    keywordstyle=\color{blue},
    commentstyle=\color{green!60!black},
    stringstyle=\color{red},
    numbers=left,
    numberstyle=\tiny\color{gray},
    stepnumber=1,
    numbersep=5pt,
    backgroundcolor=\color{verylightgray},
    showspaces=false,
    showstringspaces=false,
    showtabs=false,
    frame=single,
    tabsize=4,
    captionpos=b,
    breaklines=true,
    breakatwhitespace=false
}

% Formatage des titres
\usepackage{titlesec}
\usepackage{needspace}

\titleformat*{\section}{\needspace{5\baselineskip}\fontsize{18}{18}\bfseries\color{ece}}
\titlespacing*{\section}{0mm}{8mm}{4mm}

\titleformat*{\subsection}{\needspace{3\baselineskip}\fontsize{14}{14}\bfseries\color{ece}}
\titlespacing*{\subsection}{0mm}{6mm}{4mm}

\titleformat*{\subsubsection}{\needspace{1\baselineskip}\fontsize{12}{12}\bfseries\color{gamboge}}
\titlespacing*{\subsubsection}{0mm}{4mm}{4mm}

% En-tête et pied de page
\usepackage{fancyhdr}
\renewcommand{\headrulewidth}{0pt}

\fancypagestyle{main}{
    \fancyhead[L]{\includegraphics[width=2cm]{images/ece.png}}
    \fancyhead[R]{ING4 SE Inter}
    \fancyfoot[C]{\thepage}
}
\pagestyle{main}
    
% Page de titre
\newcommand{\HRule}[1]{\rule{\linewidth}{#1}}

% Figures
\usepackage{graphicx}
\usepackage{subcaption}
\usepackage[labelfont=bf]{caption}

% Liens
\usepackage[colorlinks=true,linkcolor=black,urlcolor=darkpowderblue]{hyperref}

\begin{document}

\title{
    \begin{figure}[htb]
        \begin{minipage}[t]{.45\textwidth}
            \centering
            \includegraphics[width=6cm]{images/ece.png}
        \end{minipage}
        \hfill
        \begin{minipage}[t]{.45\textwidth}
            \centering
            \raggedleft\vspace{-12mm}\Large{\textbf{ING4}\\ SE Inter}
        \end{minipage}  
    \end{figure}
    \vspace{2cm}
    \HRule{1.5pt} \\ [0.5cm]
    \LARGE \textbf{\Large{TP NOTÉ REPORT}\\ [5mm]
    \huge{\textcolor{ece}{Multiplayer Game Simulation with Pthreads}}}
    \HRule{1.5pt} \\ [3mm]
    
    \normalsize
    \vspace{1cm}
    \fcolorbox{white}{verylightgray}{
        \begin{minipage}{16cm}
            \vspace{0.5cm}
            \textbf{ABSTRACT} -- This report presents a concurrent multiplayer game simulation using C and pthreads. Three players move on a 5x5 grid with power-ups appearing every 3 seconds. Players battle when sharing a cell or collect power-ups to heal. The implementation demonstrates multi-threading, mutex synchronization, and collision detection.
            \vspace{0.5cm}
        \end{minipage}}
    \vfill
}

\author{
    Nolan BAYON \and Thomas MARGOTTEAU
}

\date{
    We certify that this submission is our own original work,\\
    and meets the Faculty's Expectation and Originality.\\ [2mm]
    \textbf{Paris, 05/12/2025}
}

\maketitle

\newpage
\tableofcontents

\newpage
\section{Problem Statement}

The project implements a concurrent multiplayer game with:
\begin{itemize}
    \item 3 players moving independently on a 5x5 grid (random directions, 500ms intervals)
    \item 2 power-ups spawning at random positions (3-second intervals)
    \item Player battles when sharing a cell (both lose 1-3 HP)
    \item Health regeneration from power-ups (gain 1-3 HP)
    \item Thread-safe operations using mutexes
    \item Termination when any player reaches 0 HP
\end{itemize}

\section{Sofware Architecture}

\begin{figure}[H]
    \centering
    \includegraphics[width=0.8\linewidth]{images/software_architecture.png}
    \caption{Software Architecure Diagram}
    \label{fig:placeholder}
\end{figure}
    

\section{Memory Analysis}

    \subsection{Size Command Results}
    
    \begin{verbatim}
    $ size bin/displaygamegrid
       text    data     bss     dec     hex filename
       5354     720     240    6314    18aa bin/displaygamegrid
    \end{verbatim}
    
    \begin{table}[h]
        \centering
        \begin{tabular}{|l|r|l|}
            \hline
            \textbf{Segment} & \textbf{Size (bytes)} & \textbf{Contents} \\
            \hline
            Text (code) & 5,354 & Compiled instructions \\
            \hline
            Data (initialized) & 720 & Initialized globals, string literals \\
            \hline
            BSS (uninitialized) & 240 & Uninitialized globals \\
            \hline
            \textbf{Total} & \textbf{6,314} & \textbf{Static memory} \\
            \hline
        \end{tabular}
        \caption{Memory segments analysis}
    \end{table}
    
    \subsection{Stack Usage by Function}
    
    \begin{table}[h]
        \centering
        \begin{tabular}{|l|r|l|}
            \hline
            \textbf{Function} & \textbf{Stack} & \textbf{Variables} \\
            \hline
            main() & 32B & player\_ids[3], return addr \\
            \hline
            initializePlayers() & 24B & int i, j, unique \\
            \hline
            simulateMovement() & 40B & int id, dx, dy, new\_x, new\_y \\
            \hline
            generatePowerUps() & 24B & int i, loop control \\
            \hline
            checkInteraction() & 32B & int i, j, damage, heal \\
            \hline
            displayGame() & 48B & int i, j, idx, char sym \\
            \hline
        \end{tabular}
        \caption{Stack usage per function}
    \end{table}
    
    \textbf{Calculation:} int = 4B, char = 1B, pointer = 8B (64-bit), return address = 8B.
    
    \textbf{Max call depth:} main → simulateMovement → checkInteraction = 32 + 40 + 32 = 104B
    
    \subsection{Thread Stack}
    
    Each pthread receives default stack size (2MB on Linux):
    \begin{itemize}
        \item 3 player threads: 3 × 2MB = 6MB
        \item 1 power-up thread: 1 × 2MB = 2MB
        \item Main thread: 8MB (typical)
        \item \textbf{Total thread stacks: ~16MB}
    \end{itemize}
    
    \subsection{Total Memory}
    
    \begin{table}[h]
        \centering
        \begin{tabular}{|l|r|}
            \hline
            \textbf{Category} & \textbf{Size} \\
            \hline
            Text segment & 5,354 bytes \\
            Data segment & 720 bytes \\
            BSS segment & 240 bytes \\
            Heap (malloc) & 0 bytes \\
            Thread stacks & ~16 MB \\
            \hline
            \textbf{Total} & \textbf{~16.006 MB} \\
            \hline
        \end{tabular}
        \caption{Total memory footprint}
    \end{table}
    
    Most memory (99.96\%) is pthread stack allocation. Program data is only 6.3KB.

\section{Sources}

\begin{itemize}
    \item TP Noté 2025 - Advanced C Programming. Aakash SONI, ECE Paris
    \item Advanced C Programming. Course Powerpoint Aakash SONI, ECE Paris
    \item Claude Sonnet 4.5 using Claude Code CLI
\end{itemize}

\section{Appendices}

    \subsection{Project Structure}
    
    \begin{verbatim}
    TP_Note_Advanced_C/
    ├── include/game.h
    ├── src/
    │   ├── game.c
    │   └── displaygamegrid.c
    ├── bin/displaygamegrid
    ├── obj/game.o
    ├── Makefile
    └── .gitignore
    \end{verbatim}
    
    \subsection{Compilation}
    
    \begin{verbatim}
    # Makefile compilation
    make all        # Compile
    make run        # Execute
    make clean      # Clean
    
    # Flags: -Werror -O3 -Iinclude -g
    \end{verbatim}

    \subsection{Program Output}
    \begin{figure}[H]
        \centering
        \includegraphics[width=0.5\linewidth]{images/program_output.png}
        \caption{Program running after a few iterations}
        \label{fig:placeholder}
    \end{figure}
\end{document}